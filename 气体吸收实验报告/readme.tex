%------------------------------------------------
% 导言区
\documentclass{article}
\usepackage[UTF8]{ctex}
\usepackage{hyperref}
\usepackage{enumerate}				% enumerate 有序列表环境
\usepackage{color}

\title{\heiti 关于SCU Undergraduate Thesis Template的说明}
\author{\textrm {King of Mountain}\\[0.2em]{zhaiqi9897@163.com}}
\date{\today}

%------------------------------------------------
% 正文区
\begin{document} 
	\maketitle  
	
	\begin{enumerate}
		\item 
		本模板基于四川大学软件学院2009级前辈的模板上进行调整,同时参考了\textit{latex模板及使用说明0315-1.0}中的模板
		\item
		改动之处:页面边距、行间距、论文标题的自动换行居中、各级标题字体字号、各级标题占行、附录添加、计算机代码格式设置、声明添加、致谢添加。以上改动按照2018版
		\textit{《四川大学本科毕业论文(设计)格式和参考文献著录要求》}
		\item 
		\LaTeX 突出的优点是模块化管理,虽然这份模板看上去文件比较多,但是容易上手。红字是基本流程,按照\textcolor{red}{编辑流程}所述编辑相关文件即可
		\item
		在\textcolor{blue}{scuthesis.sty}中添加了一些注释,不建议初学者改动该文件,如有需要可以参考注释改动模板
		\item 
		该模板中文摘要页的“摘要”和“关键词”没有加粗,
		可以给文档加全局选项[AutoFakeBold]进行加粗,伪粗体没有真正的粗体效果好,个人可购买商业字体使用。
		\item
		利用BibTex是规范方便的。这里提供一个方法导出BibTex的参考文献格式:打开网站\textcolor{green}{\url{https://xs.dailyheadlines.cc/}},
		输入文献信息,找到相应文献,点击下方“引用”,打开对话框后点击底部的“BibTeX”即可导出
		\item
		该模板已上传至Overleaf。
		最后推荐一个网站\textcolor{green}{\url{https://mathpix.com/}},可以导入图片、PDF等文件自动输出\LaTeX 代码
		
	\end{enumerate}

\end{document}
